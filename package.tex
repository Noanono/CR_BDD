%Figures
\usepackage{graphicx}
\usepackage{float}
\usepackage{eso-pic}

%Couleur
\usepackage{xcolor}

%Taille marges
\usepackage{geometry}
\geometry{hmargin=2.5cm,vmargin=2.5cm}

%Langue FR
\usepackage[utf8]{inputenc}
\usepackage[french]{babel}
\usepackage{lipsum}

%Ligne titre
\newcommand{\HRule}{\rule{\linewidth}{0.5mm}}

%Réglage header & footer
\usepackage{fancyhdr}
\pagestyle{fancy}
\fancyhead[L]{\rightmark}
\fancyhead[R]{}
\fancyfoot[L]{Noah \textsc{Soler}}
\fancyfoot[R]{\thepage}
\fancyfoot[C]{}
\renewcommand{\footrulewidth}{1pt}
\renewcommand{\headrulewidth}{1pt}

% table des matières par chapitre
\usepackage[french]{minitoc} 

%Pour insérer des liens
\usepackage{hyperref}

%Insérer du code
\usepackage{listings}
\usepackage{xcolor}

\definecolor{codegreen}{rgb}{0,0.6,0}
\definecolor{codegray}{rgb}{0.5,0.5,0.5}
\definecolor{codepurple}{rgb}{0.58,0,0.82}
\definecolor{backcolour}{rgb}{0.95,0.95,0.92}

\lstdefinestyle{mystyle}{
    backgroundcolor=\color{backcolour},   
    commentstyle=\color{codegreen},
    keywordstyle=\color{magenta},
    numberstyle=\tiny\color{codegray},
    stringstyle=\color{codepurple},
    basicstyle=\ttfamily\footnotesize,
    breakatwhitespace=false,         
    breaklines=true,                 
    captionpos=b,                    
    keepspaces=true,                 
    numbers=left,                    
    numbersep=5pt,                  
    showspaces=false,                
    showstringspaces=false,
    showtabs=false,                  
    tabsize=2,
    literate={à}{{\`a}}1 {è}{{\`e}}1 {é}{{\'e}}1 {ù}{{\`u}}1 {ç}{{\c{c}}}1, % Spécifiez les caractères spéciaux à inclure
    morekeywords={CREATE, INSERT, UPDATE, SELECT, VALUES, SET, WHERE, FROM, INNER JOIN, ARRAY, INTO, FOREACH, RETURN, DECLARE, BEGIN, END, LANGUAGE, RAISE, NOTICE, FOR, LOOP, TRIGGER, WHEN, ELSE, END IF, END LOOP, WHILE, IF, ELSIF, LOOP, FOR, DECLARES, RETURNRS, TEXT, LANGUAGE,RETURN},
    numbers=none % Supprime les numéros de ligne
}

\lstset{style=mystyle}

%IEEE Bib
\usepackage[style=ieee, backend=biber]{biblatex}

%intervalle d'entiers maths
\usepackage{stmaryrd}

%Notation ensembles maths
\usepackage{amssymb}

%Algorithmes
\usepackage[ruled,lined]{algorithm2e}
\SetKwInput{KwRes}{R\'esultat}%
\SetKwIF{Si}{SinonSi}{Sinon}{si}{alors}{sinon si}{sinon}{fin si}%
\SetKwFor{Tq}{tant que}{faire}{fin tq}%

%Commentaires de plusieurs lignes
\usepackage{verbatim}

%Réglages titres chapitre
\usepackage{titlesec, blindtext, color}
\definecolor{gray75}{gray}{0.75}
\newcommand{\hsp}{\hspace{20pt}}
\titleformat{\chapter}[hang]{\Huge\bfseries}{\thechapter\hsp\textcolor{gray75}{|}\hsp}{0pt}{\Huge\bfseries}

%Barres laterales
\usepackage{framed}
\renewenvironment{leftbar}{%
  \def\FrameCommand{\vrule width 0.4pt \hspace{10pt}}%
  \MakeFramed {\advance\hsize-\width \FrameRestore}}%
 {\endMakeFramed}

%Inserer une page blanche
\usepackage{afterpage}
\newcommand\emptypage{
    \null
    \thispagestyle{empty}
    \addtocounter{page}{-1}
    \newpage
    }

%Pour aligner des trucs par la droite (addition posée)
\usepackage{amsmath}

%Abastract
\newcommand{\motscles}[1]{\vspace{.2cm}\noindent{\large{\bf Mots-Clés:}} #1\\}
\newcommand{\keywords}[1]{\vspace{.2cm}\noindent{\large{\bf Keywords:}} #1\\}

\usepackage{gensymb} % Ce package fournit la commande \degree

%Afficher les subsubsection à la table des matières
\addtocounter{tocdepth}{3}
\setcounter{secnumdepth}{3}
